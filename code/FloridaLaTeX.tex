\documentclass[8pt]{article}

\title{Is Florida getting warmer?}
\author{Peter Zeng}
\date{Nov 2021}

\begin{document}
    \maketitle

    \section{Introduction}
        The goal is to answer the question: Are temperatures of one year significantly correlated with the successive year, across years in a given location? 
        In other words, we need to calculate the correlation coefficients between pairs of years
        However, the standard p-value cannot be used to calculate a correlation coefficient, because
        measurements of climatic variables in successive time-points in a time series (successive seconds, minutes, hours, months, years, etc.)
        are not independent. 
        
        In this case, we will use a permutation analysis instead, by generating a distribution of random correlation coefficients, 
        and compare the observed coefficient with this random distribution.
    
    \section{Methods}
        To find out if the correlation is significant or not, we calculate the correlation coefficient between successive years and Temperature. 
        After we get the correlation coefficient, we randomly permuting the time series 10000 times, and then recalculating the correlation
        coefficient for each randomly permuted year sequence and storing it. 

        Then we calculate what fraction of the correlation coefficients from the previous step were greater than that from the origin correlation 
        coefficient, and this will be my approximate p-value

    \section{Results}
        The correlation coefficient between temperatures and years is 0.3261, and the fraction of the correlation coefficients which   
        were greater than the origin correlation coefficient is 1e-04, which is our p-value. In this case, we can say that there is a statistically signigcant correlation 
        between years and temperation, in other words, temperatures are significantly correlated with years.

\end{document}